\documentclass{article}
\usepackage[fleqn]{amsmath} % gather,gather*,bmatrix
\usepackage{mathtools} % multilined
\begin{document}

Factory production problem from https://www.facebook.com/groups/mathhelp/posts/5371851856171657/

References:

https://www.youtube.com/watch?v=LE9Lnl2iXrA\\
https://www.youtube.com/watch?v=woJAb5EgjtI\\
https://www.pmcalculators.com/simplex-method-calculator/\\
https://www.youtube.com/watch?v=X5db8RXahaI\\


This can be construed as a linear programming problem.  Let a, b, c, and d be the hours allotted to factories A, B, C, and D.
\\
\\
Minimize:
\begin{equation*}
\begin{aligned}
a + b + c + d
\end{aligned}
\end{equation*}

subject to:
\begin{equation*}
\begin{aligned}
10a + 10b + 10c + 10d \geq 3000\\
10a + 20b + 20d \geq 5000\\
20a + 40c + 40d \geq 6000\\
a, b, c, d \geq 0\\
\end{aligned}
\end{equation*}

We convert this to standard form by adding surplus variables $s_i$:
\\
\\
Minimize:
\begin{equation*}
\begin{aligned}
a + b + c + d
\end{aligned}
\end{equation*}

subject to:
\begin{equation*}
\begin{aligned}
10a + 10b + 10c + 10d - s_1 = 3000\\
10a + 20b + 20d - s_2 = 5000\\
20a + 40c + 40d - s_3 = 6000\\
a, b, c, d \geq 0\\
s_1, s_2, s_3 \geq 0\\
\end{aligned}
\end{equation*}

To use the simplex method, create the initial tableau.
Note we will maximize the negation of the original objection function:
\begin{equation*}
\begin{aligned}
Z = -(a + b + c + d)\\
Z + a + b + c + d = 0
\end{aligned}
\end{equation*}

\[
	\begin{bmatrix}
	a & b & c & d & s_1 & s_2 & s_3 & Z\\
	10 & 10 & 10 & 10 & -1 &  0 & 0  & 0 & 3000\\
	10 & 20 & 0  & 20 &  0 & -1 & 0  & 0 & 5000\\
	20 &  0 & 40 & 40 &  0 &  0 & -1 & 0 & 6000\\
	1  &  1 &  1 &  1 &  0 &  0 & 0  & 1 & 0
	\end{bmatrix}
\]


I'm too lazy to actually run the algorithm, so I put it into a solver.  The results are here:  https://tinyurl.com/wb7x3wb2

Looks like my solution to run factory D for 300 hours is as good as their solution to run B for 150, C for 50, and D for 100.  However, the products will be done in 150 hours since the factories may run in parallel.



\end{document}